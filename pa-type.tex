\begin{frame}{Fundamentals}
    \begin{defn}
        Let $G \leq \sym \Omega$ be transitive.
        $G$ is \emph{primitive} if
        \\
        there exists no non-trivial
        $G$-invariant partition of $\Omega$.
    \end{defn}
    \pause

    \begin{defn}
        Let $G \leq \sym \Omega$ and $H \leq \sym \Delta$.
        We call $(f, \varphi)$ with
        $f \from \Omega \iso \Delta$
        and
        $\varphi \from G \iso H$
        a \emph{permutation isomorphism}
        if
        \\
        \pause
        $\forall g \in G ~ \forall \alpha \in \Omega :$
        \[
            f(\alpha ^ g) = f(\alpha) ^ {\varphi(g)}.
        \]
    \end{defn}
\end{frame}

\note[itemize]{
\item
Explain perm iso:
\begin{itemize}
    \item
    Relabel points and how to map $G \to H$ accordingly.
    \item
    $G, H \leq \sym \Omega$ perm iso iff. exists $\sigma \in \sym \Omega$ with
    $G ^ \sigma = H$.
    \item
    $f \from \Omega \iso \Delta$ induces unique
    group hom
    $\sym \Omega \iso \sym \Delta$.
\end{itemize}
\item
perm iso can be extended to the whole symmetric group
\\
$\leadsto$
The normaliser of $G$ over $\Omega$ is mapped to the normaliser of $H$ over
$\Delta$.
}

\begin{frame}{Socles}
    \begin{defn}
        Let $G$ be a group. The \emph{socle} of $G$, denoted $\soc G$,
        is the group generated by all minimal normal subgroups of $G$.
    \end{defn}
    \pause

    \begin{thm}
        The socle of a primitive group is characteristically simple.
    \end{thm}
    \pause

    \begin{thm}[O'Nan-Scott]
        Let $G \leq \sym \Omega$ be primitive.
        We know all possible permutational isomorphism types of
        \pause
        \vspace{-0.5em}
        \begin{itemize}
            \item
            $\soc G$,
            \pause
            \item
            $N_{\sym \Omega}(\soc G)$.
        \end{itemize}
    \end{thm}
\end{frame}

\begin{frame}{The AS Type}
    \begin{defn}
        Let $G \leq \sym \Omega$ be a primitive group.
        \\
        $G$ is a group of \emph{AS type} if
        \vspace{-0.5em}
        \pause
        \begin{itemize}
            \item
            $G$ is almost simple,
            \pause
            \item
            $\soc G$ is non-abelian simple and non-regular.
        \end{itemize}
    \end{defn}
\end{frame}

\note{
    Explain AS via Normaliser of $T$.
}

\begin{frame}{Wreath Products (1)}
    \begin{defn}
        Let $H$ be a group and let $K \leq S_\ell$.
        $K$ acts on $H ^ \ell$ by permuting components.
        \pause
        The group
        $H \wr K := H ^ \ell \rtimes K$
        is the \emph{wreath product of $H$ with $K$}.
%        \\
%        \pause
%        $H ^ \ell$ is called the \emph{base group}.
%        \pause
%        $K$ is called the \emph{top group}.
    \end{defn}
%    \vspace{1em}
%    \pause
%    \begin{thm}
%        $\aut( T ^ \ell ) \cong \aut(T) \wr S_\ell$.
%    \end{thm}
\end{frame}

\note{
    Make sure to explain intuition behind wreath products
}

\begin{frame}{Wreath Products (2)}
    \begin{defn}
        Let $H \leq \sym \Delta$ and $K \leq S_\ell$.
        The base group $H ^ \ell$ acts component-wise on $\Delta ^ \ell$.
        The top group $K$ acts on $\Delta ^ \ell$ by permuting the components.
        \pause
        This yields the \emph{product action of $H \wr K$} on $\Delta ^ \ell$.
    \end{defn}
\end{frame}

\begin{frame}{The PA Type}
    \begin{defn}
        Let $G \leq \sym \Omega$ be a primitive group.
        $G$ is a group of \emph{PA type} if
        \\[-0.5em]
        \pause
        \[
        G \circlearrowright \Omega
        \iso
        \myhat G \circlearrowright \Delta ^
        \ell
        \]
        with:
        \vspace{-0.5em}
        \pause
        \begin{itemize}
            \item
            $T \circlearrowright \Delta$,
            \pause
            \item
            $\soc \myhat G = T ^ \ell$ in component-wise action,
            \pause
            \item
            $\myhat G \leq N_{\sym \Delta}(T) \wr S_\ell$ in product action.
        \end{itemize}
    \end{defn}
    \pause

    \begin{lemma}
        Let $T ^ \ell \leq \sym \Delta ^ \ell$ act component-wise,
        transitively, and non-regularly. Then
        \\[-1.0em]
        \pause
        \[
        N_{\sym \Delta ^ \ell}( T ^ \ell ) = N_{\sym \Delta}(T) \wr S_\ell.
        \]
    \end{lemma}
\end{frame}

\note[itemize]
{
\item
PA: an example is our running example!!
\item
\"Uberleitung to key idea, sketch on black board!!: \\
$\soc G \text{ char } G$ \\
$\soc G \unlhd N(G)$ \\
Thus contained in normaliser of socle
}

\begin{frame}{The Key Idea ...}
    \centering
    {\Large
    Construct $N_{\sym \Omega}(\soc G)$!
    }
\end{frame}

\begin{frame}{... And Why It Works ...}
    \begin{lemma}
        Let $G \leq \sym \Omega$ be primitive of type PA.
        Then
        \vspace{-0.5em}
        \[
            [ N_{\sym \Omega}(\soc G) : \soc G ]
            \leq \sqrt n \cdot 2 ^ {\log n \log \log n}.
        \]
    \end{lemma}
    \pause

    \begin{lemma}
        Let $G = \gen X \leq \sym \Omega$ be primitive of type PA.
        Let
        $N_{\sym \Omega}(\soc G) = \gen Y$ be known.
        \\
        \pause
        Then $N_{\sym \Omega}(G)$ can be computed in time
        \vspace{-0.5em}
        \[
            O(n ^ 3 \cdot 2 ^ {2 \log n \log \log n} \cdot \abs X).
        \]
    \end{lemma}
\end{frame}

\note{
Explain:
\\
$\abs{\out T} \leq \sqrt n$
\\
$\abs {S_\ell} \leq \ell ^ \ell$
}

\begin{frame}{... And How To Do It}
    Compute:
    \[
        \soc G \circlearrowright \Omega
        \iso
        T ^ \ell \circlearrowright \Delta ^ \ell
    \]
\end{frame}

\note[itemize]{%
\item
equal and not only isomorphic
\item
PA WP is a very very special group!
}
