\begin{frame}{Fundamentals}
    \begin{defn}
        FIXME primitive
    \end{defn}
    \begin{defn}
        FIXME perm iso
    \end{defn}
    \begin{rem}
        FIXME
        $f \from \Omega \iso \Delta$ induces unique
        group hom
        $\sym \Omega \iso \sym \Delta$.
    \end{rem}
\end{frame}

\note{
Explain perm iso:
FIXME
}

\begin{frame}{Socles}
    \begin{defn}
        Let $G$ be a group. The \emph{socle} of $G$, denoted $\soc G$,
        is the group generated by all minimal normal subgroups of $G$.
    \end{defn}
    \pause

    \begin{thm}
        The socle of a primitive group is characteristically simple.
    \end{thm}
    \pause

    \begin{thm}[O'Nan-Scott]
        Let $G \leq \sym \Omega$ be primitive.
        All possible permutational isomorphism types of $\soc G$ and
        $N_{\sym \Omega}(\soc G)$ are known.
    \end{thm}
\end{frame}

\begin{frame}{Wreath Products (1)}
    \begin{defn}
        Let $H \leq \sym \Delta$ and $K \leq S_\ell$.
        $K$ acts on the components of $H ^ \ell$.
        \pause
        The semidirect product
        $H \wr K = H ^ \ell \rtimes K$
        is called the \emph{wreath product of $H$ with $K$}.
    \end{defn}
    \vspace{1em}
    \pause

    \begin{thm}
        $\aut( T ^ \ell ) \cong \aut(T) \wr S_\ell$
    \end{thm}
\end{frame}

\note{
    Make sure to explain intuition behind wreath products
}

\begin{frame}{Wreath Products (2)}
    \begin{defn}
        FIXME Product action
    \end{defn}
    \vspace{1em}
    \pause

    \begin{thm}
        Let $H \leq \sym \Delta$ and $K \leq S_\ell$.
        $H \wr K$ in product action is primitive
        if and only if
        $H$ is primitive and non-regularly and $K$ is transitive.
    \end{thm}
\end{frame}

\note{
Explain WP actions via

base

top
}

\begin{frame}{The AS Type}
    \begin{defn}
        Let $G \leq \sym \Omega$ be a primitive group.
        \\
        We say $G$ is a group of \emph{AS type} if
        $\soc G = T$ is non-abelian simple
        and $G$ is almost simple.
    \end{defn}
\end{frame}

\begin{frame}{The PA Type}
    \begin{defn}
        Let $G \leq \sym \Omega$ be a primitive group.
        \\
        We say $G$ is a group of \emph{PA type} if it is permutation
        isomorphic to a group $\myhat G \leq \sym \Delta ^ \ell$ with:
        \vspace{-0.5em}
        \pause
        \begin{itemize}
            \item
            $\soc \myhat G = T ^ \ell$,
            \pause
            \item
            $\myhat G \leq N_{\sym \Delta}(T) \wr S_\ell$.
        \end{itemize}
    \end{defn}

    \begin{lemma}
        $N_{\sym \Delta ^ \ell}( T ^ \ell ) = N_{\sym \Delta}(T) \wr S_\ell$.
    \end{lemma}
\end{frame}

\note[itemize]
{
\item
AS: explain relationship with normaliser $N_{\sym \Delta}(T)$.
\item
\"Uberleitung to key idea: \\
$\soc G \text{ char } G$ \\
$\soc G \unlhd N(G)$ \\
Thus contained in normaliser of socle
}

\begin{frame}{The Key Idea ...}
    \centering
    {\Large
    Construct $N_{\sym \Omega}(\soc G)$!
    }
\end{frame}

\begin{frame}{... And Why It Works ...}
    \begin{lemma}
        Let $G \leq \sym \Omega$ be primitive of type PA.
        Then
        \vspace{-0.5em}
        \[
            [ N_{\sym \Omega}(\soc G) : \soc G ]
            \leq \sqrt n \cdot 2 ^ {\log n \log \log n}.
        \]
    \end{lemma}
    \pause

    \begin{lemma}
        Let $G = \gen X \leq \sym \Omega$ be primitive of type PA.
        Furthermore let a generating set for
        $N_{\sym \Omega}(\soc G)$ be known.
        \\
        \pause
        Then $N_{\sym \Omega}(G)$ can be computed in time
        \vspace{-0.5em}
        \[
            O(n ^ 3 \cdot 2 ^ {2 \log n \log \log n} \cdot \abs X).
        \]
    \end{lemma}
\end{frame}

\note{
Explain $\abs{\out T} \leq \sqrt n$.
}

\begin{frame}{... And How To Do It}
    Compute:
    \[
        \soc G \circlearrowright \Omega
        \iso
        T ^ \ell \circlearrowright \Delta ^ \ell
    \]
    \pause

    Then:
    \begin{align*}
        G
        ~
        \lhook\joinrel\longrightarrow
        & ~ N_{\sym \Delta ^ \ell}(T ^ \ell)
        \\
        & =
        N_{\sym \Delta}(T) \wr S_\ell
    \end{align*}
\end{frame}

\note[itemize]{%
\item
equal and not only isomorphic
\item
PA WP is a very very special group!
}
