\begin{frame}{Complexity Classes}
    We use big $O$ notation.
    \pause
    \\[2em]
    \hspace{2em}
    \begin{tabular}{l l}
    Polynomial Time:
    &
    $f \in O(n ^ c)$
    \\
    \pause
    Quasipolynomial Time:
    &
    $f \in 2 ^ {O((\log n) ^ c)}$
    \\
    \pause
    Simply Exponential Time:
    &
    $f \in 2 ^ {O(n)}$
    \\
    \pause
    Exponential Time:
    &
    $f \in 2 ^ {O(n ^ c)}$
    \end{tabular}
\end{frame}

\note[enumerate]{
\item
properly explain why quasipolynomial is not polynomial!
\item
changing input size to $\log n$ jumps up two classes
\item
Normaliser is harder than
\\
String-Iso, Centraliser, Intersection
\\
Graph-Iso
}

\begin{frame}{Normaliser in the Symmetric Group}
    \centering
    \Large
    Why restrict to $N_{\sym \Omega}(G)$?
\end{frame}

\begin{frame}{Normaliser and Subproblems}
    \begin{tabular}{l l}
        \multicolumn{2}{c}
        {\underline{Simply Exponential} \hspace{4em} $~$}
        \\[0.5em]
        \multicolumn{2}{c}
        {Normalisers of arbitrary groups}
        \\[2.5em]
        \pause

        \multirow{2}{*}{\parbox{0.5\linewidth}{
            \underline{Polynomial}
            \\[0.5em]
            Normalisers of groups with \\ restricted composition factors
            \\[0.5em]
            \pause
            \hspace{-0.7em}
            Normalisers of simple groups
        }}
        \pause
        &
        \multirow{2}{*}{\parbox{0.5\linewidth}{
            \underline{Quasipolynomial}
            \\[0.5em]
            {Normalisers of primitive \\ groups}
        }}
    \end{tabular}
    \vfill
\end{frame}
