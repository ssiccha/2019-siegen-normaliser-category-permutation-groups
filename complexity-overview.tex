\begin{frame}{Complexity Classes}
    We use big $O$ notation.
    \pause
    \\[2em]
    \hspace{2em}
    \begin{tabular}{l l}
    Polynomial Time:
    &
    $f \in O(n ^ c)$
    \\
    \pause
    Quasipolynomial Time:
    &
    $f \in 2 ^ {O((\log n) ^ c)}$
    \\
    \pause
    Simply Exponential Time:
    &
    $f \in 2 ^ {O(n)}$
    \\
    \pause
    Exponential Time:
    &
    $f \in 2 ^ {O(n ^ c)}$
    \end{tabular}
\end{frame}

\note[enumerate]{
\item
properly explain why quasipolynomial is not polynomial!
\item
changing input size to $\log n$ jumps up two classes
\item
$A$ is easier than $B$
\\ OR \\
$A$ can be embedded into $B$
}

\begin{frame}{Complexity Overview}
    \begin{tabular}{l l}
    \only<4->{%
        \underline{Simply Exponential:}
    }
    \only<-3>{\hphantom{%
        \underline{Simply Exponential:}
    }}
    &
    \\
    &
    \only<4->{%
    Normaliser
    }
    \only<-3>{\hphantom{%
    Normaliser
    }}
    \\[1em]
    \only<2->{%
    \underline{Quasipolynomial:}
    }
    \only<-1>{\hphantom{%
    \underline{Quasipolynomial:}
    }}
    &
    \\
    &
    \only<3->{%
    String-Iso, Intersection, Centraliser
    }
    \only<-2>{\hphantom{%
    String-Iso, Intersection, Centraliser
    }}
    \\[2em]
    &
    \only<2->{%
    Graph-Iso
    }
    \only<-1>{\hphantom{%
    Graph-Iso
    }}
    \\[1em]
    \underline{Polynomial:}
    &
    \\
    &
    Base \& SGS, Composition Series,
    Socle
    \\
    \end{tabular}
\end{frame}

\note{
FIXME: USE UNCOVER OR ONLY ALTERNATIVE
https://tex.stackexchange.com/questions/13793/beamer-alt-command-like-visible-instead-of-like-only
}

\begin{frame}{Normaliser and Subproblems}
    \begin{tabular}{l l}
        \multicolumn{2}{c}
        {\underline{Simply Exponential} \hspace{4em} $~$}
        \\[0.5em]
        \multicolumn{2}{c}
        {Normalisers of arbitrary groups}
        \\[2.5em]
        \pause

        \multirow{2}{*}{\parbox{0.5\linewidth}{
            \underline{Polynomial}
            \\[0.5em]
            Normalisers of groups with \\ restricted composition factors
            \\[0.5em]
            Normalisers of simple groups
        }}
        \pause
        &
        \multirow{2}{*}{\parbox{0.5\linewidth}{
            \underline{Quasipolynomial}
            \\[0.5em]
            {Normalisers of primitive \\ groups}
        }}
    \end{tabular}
    \vfill
\end{frame}

\note{
Now also explain why we're not precise about whether $N_{sym \Omega}(G)$ or
$N_H(G)$.
}
