\begin{frame}{Permutation Homomorphisms (1)}
    \begin{defn}
        Let $G \leq \sym \Omega$ and $H \leq \sym \Delta$ be permutation groups.
        \pause
        Let
        $f \from \Omega \to \Delta$
        be a map
        and
        $\varphi \from G \to H$
        be a group hom..
        \\
        \pause
        The pair $(f, \varphi)$ is called a
        \emph{%
        permutation hom. from $(G, \Omega)$ to $(H, \Delta)$
        }
        if for all $g \in G$ holds:
        \pause
        \[
        \begin{tikzcd}[ampersand replacement=\&]
            \Omega
                \ar[r, "g"]
                \ar[d, "f"]
            \&
            \Omega
                \ar[d, "f"]
            %
            \\
            %
            \Delta
                \ar[r, "\varphi(g)"]
            \&
            \Delta
        \end{tikzcd}
        \]
    \end{defn}
\end{frame}

\begin{frame}{Permutation Homomorphisms (2)}
    \[
    \begin{tikzcd}[ampersand replacement=\&]
        \Omega
            \ar[r, "g"]
            \ar[d, "f"]
        \&
        \Omega
            \ar[d, "f"]
        %
        \\
        %
        \Delta
            \ar[r, "\varphi(g)"]
        \&
        \Delta
    \end{tikzcd}
    \]

    \begin{rem}
        Let $G \leq \sym \Omega$ and $H \leq \sym \Delta$.
        The map $f \from \Omega \onto \Delta$ uniquely determines,
        if it exists,
        a permutation homomorphism
        $(f, \varphi)$.
    \end{rem}
\end{frame}

\begin{frame}{Permutation Homomorphisms (3)}
    \begin{lemma}
        Let $G \leq \sym \Omega$ and $f \from \Omega \to \Delta$.
        There exists a permutation hom.
        $(f, \varphi)$
        if and only if
        \pause
        \[
            \sset{ f ^ {-1}(\set{x}) }{ x \in \im f }
        \]
        is $G$-invariant.
    \end{lemma}
\end{frame}

\note{
    Give examples:
    \\
    map to orbits
    \\
    factor out block systems
    \\
    if $f$ is bijection such a $\varphi$ always exists.
}

\begin{frame}{PermGrp}
    \begin{defn}
        The \emph{category of permutation groups},
        denoted $\permgrp$,
        consists of all pairs $(G, \Omega)$ with $G \leq \sym \Omega$ as objects
        with permutation homomorphisms as morphisms.
    \end{defn}
\end{frame}

\note{equiv to cat of $(G, \Omega, \rho)$}

\begin{frame}{Product in PermGrp}
    \begin{lemma}
        Let $G \leq \sym \Omega$ and $H \leq \sym \Delta$ be permutation
        groups.
        Then $(G \times H, \Omega \times \Delta)$ with
        $(p_1, \pi_1)$ and $(p_2, \pi_2)$ is a product in $\permgrp$.
    \end{lemma}
\end{frame}

\begin{frame}{Cartesian Decompositions}
    \begin{defn}
        Let $\mathcal C$ be a category and $X$ an object of $\mathcal C$.
        A family of morphisms $(f_i)_{i \in I}$ with
        $f_i \from X \to X_i$
        is called a
        \emph{cartesian decomposition of $X$}
        if
        \pause
        \[
            \prod_{i \in I} f_i
            \from
            X
            \to
            \prod_{i \in I} X_i
        \]
        is an isomorphism.
    \end{defn}
    \pause

    \begin{lemma}
        %Let $\mathcal C$ be a category and $X$ an object of $\mathcal C$.
        A family $(f_i)_{i \in I}$ is a cartesian decomposition of $X$
        if and only if
        $X$ with $(f_i)_{i \in I}$ forms a product in $C$.
    \end{lemma}
\end{frame}

\note[itemize]
{
    \item
    explain cartesian decomposition of sets and of perm groups
    \item
    mention combinatorial cartesian decompositions by
    Laszlo Kovacs, Cheryl Praeger and Csaba Schneider!
    \\
    their theorem
    \\
    cartesian decompositions are to PA WPs what
    block systems are to IMP WPs.
    \item
    cartesian decomposition on sets, groups and to deconstruct $\soc G$
}

\begin{frame}{Homogeneous Cartesian Decompositions}
    \begin{defn}
        Let $(f_i)_{i \in I}$ be a cartesian decomposition of $X$.
        We call $(f_i)_{i \in I}$ a
        \emph{homogeneous cartesian decomposition of $X$}
        if
        \\
        \pause
        for all $i,j \in I$ we have $f_i(X) \cong f_j(X)$.
    \end{defn}
    \pause

    \begin{defn}
        Let $(f_i)_{i \in I}$ be a cartesian decomposition of $X$.
        We call $(f_i)_{i \in I}$ a
        \emph{strongly homogeneous cartesian decomposition of $X$}
        if
        \\
        \pause
        for all $i,j \in I$ we have $f_i(X) = f_j(X)$.
    \end{defn}
    \pause

    $\leadsto$
    Compute a strongly homogeneous cartesian decomposition of the permutation
    group $\soc G$!
\end{frame}

\note
{
Explain hom cartesian decomposition and str hom cartesian decomposition of
$\Omega$ and then of $\soc G$.
}
