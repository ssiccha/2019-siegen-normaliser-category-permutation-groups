\documentclass{beamer}
\usetheme{metropolis}           % Use metropolis theme

%\setbeameroption{hide notes}
\setbeameroption{show notes}
%\setbeameroption{show notes on second screen}

% Packages
% Maths
\usepackage{amsthm}

% Package Settings
% AMSTHM
\theoremstyle{plain}
\newtheorem{thm}{Theorem}[section]
\newtheorem{mylemma}[thm]{Lemma}
\newtheorem{cor}[thm]{Corollary}
\newtheorem{rem}[thm]{Remark}

\theoremstyle{definition}
\newtheorem{defn}[thm]{Definition}

% BEAMER
% Hack for only and tabular
\newcommand{\tabularonly}{AS}
% Hack for tilde and hat
\newcommand{\mytilde}[1]    {\ensuremath{\tilde{#1}}}
\newcommand{\myhat}[1]    {\ensuremath{\widehat{#1}}}
% Hack to display \Gamma with a little bit of space to the left
% Otherwise \ind _ \Gamma ^ H looks really weird
\let\oldGamma\Gamma
\renewcommand{\Gamma}{{\hspace{0.5pt} \oldGamma}}
% Hack to reference the notation-definition in "lift a.s. normaliser"
\newcommand{\refdef}    {{(6.1)}}

% Symbols
% Write function definitions like
% f \from A \to B
\newcommand{\from}	{\ensuremath{\colon}}
\renewcommand{\to}	{\ensuremath{\rightarrow}}
\newcommand{\into}	{\ensuremath{\hookrightarrow}}
\newcommand{\iso}	{\ensuremath{\xrightarrow{\,\raisebox{-1pt}{\ensuremath{\scriptstyle{\sim}}}\,}}}
\newcommand{\N}	{\ensuremath{\mathbb N}}
\newcommand{\Z}	{\ensuremath{\mathbb Z}}
\newcommand{\Q}	{\ensuremath{\mathbb Q}}
\newcommand{\Sn}{{\ensuremath{\mathrm{S}_n}}}
\newcommand{\uln} {{\ensuremath{\underline n}}}

%%% Unary Operators
% Functions
\DeclareMathOperator{\dom}{dom}
\DeclareMathOperator{\id}{id}
\DeclareMathOperator{\im}{Im}
% Groups
\DeclareMathOperator{\sym}{Sym}
\DeclareMathOperator{\alternating}{Alt}
\DeclareMathOperator{\soc}{soc}
\DeclareMathOperator{\aut}{Aut}
\DeclareMathOperator{\out}{Out}
\DeclareMathOperator{\ind}{Ind}
\DeclareMathOperator{\res}{Res}
\DeclareMathOperator{\stab}{Stab}
\newcommand{\hatOperator}   {\ensuremath{\, \myhat \cdot\,}}
\def\Norm#1#2{\mathrm{N}_{#1}(#2)}

%%% Binary Operators %%%
\newcommand{\union}{\mathbin{\cup}}
\newcommand{\bigunion}{\mathbin{\bigcup}}
\newcommand{\disjointunion}{\mathbin{\uplus}}
\newcommand{\intersection}{\mathbin{\cap}}
\newcommand{\bigintersection}{\mathbin{\bigcap}}
\newcommand{\norm}{\mathbin{\lhd}}
\newcommand{\characteristic}{\mathbin{\text{char}}}


% New commands
\newcommand{\abs}[1]	{%
	\ensuremath{
		\left| #1 \right|
	}%
}
\newcommand{\gen}[1]	{
	\ensuremath{
		\left\langle \, #1 \, \right\rangle
	}
}
\newcommand{\set}[1]	{
	\ensuremath{
		\left\{ #1 \right\}
	}
}
\newcommand{\sset}[2] {
	\ensuremath{
		\left\{ \left.\,
			#1
		~\right|~
			#2
		\,\right\}
	}
}


\title{Normalisers in Quasipolynomial Time and \\
the Category of Permutation Groups}
\date{May 9, 2019}
\author{Sergio Siccha}
\institute{Lehrstuhl B f\"ur Mathematik, RWTH Aachen}

\begin{document}
\maketitle
\section{Introduction}
\begin{frame}{Conventions}
\begin{itemize}
\setlength\itemsep{\fill}
\item
$\log = \log_2$.
\item
All groups and sets are finite!
\item
$\Omega, \Delta, \Gamma$ sets,
$G \leq \sym \Omega$, $g \in G$, $\alpha \in \Omega$.
\item
Functions from the left $f(x)$
but group actions from the right: $\alpha ^ g = g(\alpha)$.
    \begin{itemize}
        \item
        $G$ acts on functions $\Omega \to \Delta$ via
        $f ^ g = f \circ g ^ {-1}$.
    \end{itemize}
\item
$T$ \emph{always} denotes a finite non-abelian simple group.
If $T \leq \sym \Delta$ it acts transitively and non-regularly on $\Delta$.
\end{itemize}
\end{frame}

\begin{frame}{Goal}
    \begin{thm}
        Let $G = \gen{X} \leq \sym \Omega$ be a primitive group
        of PA type.
        The normaliser
        $N_{\sym \Omega}(G)$
        can be computed in quasipolynomial time
        $O(n ^ 3 \cdot 2 ^ {2 \log n \log \log n} \cdot \abs X)$.
    \end{thm}
\end{frame}

\begin{frame}{Recursion}
    \begin{center}
        \hspace{-5em}
        \begin{tabular}{r c}
            & Intransitive
            \\
            Mun See Chang & $\updownarrow$
            \\
            & Transitive
            \\
            & $\updownarrow$
            \\
            & Primitive
            \\
            Me & $\updownarrow$
            \\
            & Simple
        \end{tabular}
    \end{center}
\end{frame}


\section{Some Problems in Computational Group Theory}
\begin{frame}{Complexity Classes}
    Big $O$ Notation
    \pause
    \\[2em]
    \hspace{2em}
    \begin{tabular}{l l}
    Polynomial Time:
    &
    $f \in O(n ^ c)$
    \\
    \pause
    Quasipolynomial Time:
    &
    $f \in 2 ^ {O((\log n) ^ c)}$
    \\
    \pause
    Simply Exponential Time:
    &
    $f \in 2 ^ {O(n)}$
    \\
    \pause
    Exponential Time:
    &
    $f \in 2 ^ {O(n ^ c)}$
    \end{tabular}
    \pause
    \\[2em]
    We say a problem $A$ is \emph{polynomial time reducible} to a problem $B$
    if there exists a polynomial time algorithm that transforms
    \vspace{-0.5em}
    \pause
    \begin{itemize}
        \item
        instances of $A$ into instances of $B$, and
        \pause
        \item
        solutions of $B$ into solutions of $A$.
    \end{itemize}
\end{frame}

\note[enumerate]{
\item
properly explain why quasipolynomial is not polynomial!
\item
changing input size to $\log n$ jumps up two classes
\item
$A$ is easier than $B$
\\ OR \\
$A$ can be embedded into $B$
}

\begin{frame}{Complexity Overview}
    \begin{tabular}{l l}
    \only<4->{%
        \underline{Simply Exponential:}
    }
    \only<-3>{\hphantom{%
        \underline{Simply Exponential:}
    }}
    &
    \\
    &
    \only<4->{%
    Normaliser
    }
    \only<-3>{\hphantom{%
    Normaliser
    }}
    \\[1em]
    \only<2->{%
    \underline{Quasipolynomial:}
    }
    \only<-1>{\hphantom{%
    \underline{Quasipolynomial:}
    }}
    &
    \\
    &
    \only<3->{%
    String-Iso, Intersection, Centraliser
    }
    \only<-2>{\hphantom{%
    String-Iso, Intersection, Centraliser
    }}
    \\[2em]
    &
    \only<2->{%
    Graph-Iso
    }
    \only<-1>{\hphantom{%
    Graph-Iso
    }}
    \\[1em]
    \underline{Polynomial:}
    &
    \\
    &
    Base \& SGS, Composition Series,
    Socle
    \\
    \end{tabular}
\end{frame}

\note{
FIXME: USE UNCOVER OR ONLY ALTERNATIVE
https://tex.stackexchange.com/questions/13793/beamer-alt-command-like-visible-instead-of-like-only
}

\begin{frame}{Normaliser and Subproblems}
    \begin{tabular}{l l}
        \multicolumn{2}{c}
        {\underline{Simply Exponential} \hspace{4em} $~$}
        \\[0.5em]
        \multicolumn{2}{c}
        {Normalisers of arbitrary groups}
        \\[2.5em]
        \pause

        \multirow{2}{*}{\parbox{0.5\linewidth}{
            \underline{Polynomial}
            \\[0.5em]
            Normalisers of groups with \\ restricted composition factors
            \\[0.5em]
            Normalisers of simple groups
        }}
        \pause
        &
        \multirow{2}{*}{\parbox{0.5\linewidth}{
            \underline{Quasipolynomial}
            \\[0.5em]
            {Normalisers of primitive \\ groups}
        }}
    \end{tabular}
    \vfill
\end{frame}


\section{PA Type Groups and How To Normalise Them}
\begin{frame}{Fundamentals}
    \begin{defn}
        Let $G \leq \sym \Omega$ be transitive.
        $G$ is called \emph{primitive} if there exists no non-trivial
        $G$-invariant partition of $\Omega$.
    \end{defn}
    \pause

    \begin{defn}
        Let $G \leq \sym \Omega$ and $H \leq \sym \Delta$ be permutation groups.
        We call a pair $(f, \varphi)$ with
        $f \from \Omega \to \Delta$
        and
        $\varphi \from G \to H$
        a \emph{permutation isomorphism}
        if
        \pause
        for all
        $g \in G$ and $\alpha \in \Omega$ holds
        $f(\alpha ^ g) = f(\alpha) ^ {\varphi(g)}$.
    \end{defn}
\end{frame}

\note[itemize]{
\item
Explain perm iso:
\begin{itemize}
    \item
    Relabel points and how to map $G \to H$ accordingly.
    \item
    $G, H \leq \sym \Omega$ perm iso iff. exists $\sigma \in \sym \Omega$ with
    $G ^ \sigma = H$.
    \item
    $f \from \Omega \iso \Delta$ induces unique
    group hom
    $\sym \Omega \iso \sym \Delta$.
\end{itemize}
\item
perm iso can be extended to the whole symmetric group
\\
$\leadsto$
The normaliser of $G$ over $\Omega$ is mapped to the normaliser of $H$ over
$\Delta$.
}

\begin{frame}{Socles}
    \begin{defn}
        Let $G$ be a group. The \emph{socle} of $G$, denoted $\soc G$,
        is the group generated by all minimal normal subgroups of $G$.
    \end{defn}
    \pause

    \begin{thm}
        The socle of a primitive group is characteristically simple.
    \end{thm}
    \pause

    \begin{thm}[O'Nan-Scott]
        Let $G \leq \sym \Omega$ be primitive.
        All possible permutational isomorphism types of $\soc G$
        \pause
        and
        $N_{\sym \Omega}(\soc G)$ are known.
    \end{thm}
\end{frame}

\begin{frame}{Wreath Products (1)}
    \begin{defn}
        Let $H \leq \sym \Delta$ and $K \leq S_\ell$.
        $K$ acts on the components of $H ^ \ell$.
        \pause
        The semidirect product
        $H \wr K = H ^ \ell \rtimes K$
        is called the \emph{wreath product of $H$ with $K$}.
        \pause
        $H ^ \ell$ is called the \emph{base group}.
        \pause
        $K$ is called the \emph{top group}.
    \end{defn}
    \vspace{1em}
    \pause
    \begin{thm}
        Let $T ^ \ell \leq \sym \Delta ^ \ell$ act component-wise.
        Then $\aut( T ^ \ell ) \cong \aut(T) \wr S_\ell$.
    \end{thm}
\end{frame}

\note{
    Make sure to explain intuition behind wreath products
}

\begin{frame}{Wreath Products (2)}
    \begin{defn}
        Let $H \leq \sym \Delta$ and $K \leq S_\ell$.
        The base group $H ^ \ell$ acts component-wise on $\Delta ^ \ell$.
        The top group $K$ acts on the components of $\Delta ^ \ell$.
        \pause
        This yields the \emph{product action of $H \wr K$} on $\Delta ^ \ell$.
    \end{defn}
\end{frame}

\note{
Explain WP actions via

base

top
}

\begin{frame}{The AS Type}
    \begin{defn}
        Let $G \leq \sym \Omega$ be a primitive group.
        \\
        We say $G$ is a group of \emph{AS type} if
        $\soc G = T$ is non-abelian simple and non-regular,
        and $G$ is almost simple.
    \end{defn}
\end{frame}

\note{
    Explain AS via Normaliser of $T$.
}

\begin{frame}{The PA Type}
    \begin{defn}
        Let $G \leq \sym \Omega$ be a primitive group.
        \\
        We say $G$ is a group of \emph{PA type} if it is permutation
        isomorphic to a group $\myhat G \leq \sym \Delta ^ \ell$ with:
        \vspace{-0.5em}
        \pause
        \begin{itemize}
            \item
            $\soc \myhat G = T ^ \ell$ in component-wise action,
            \pause
            \item
            $\myhat G \leq N_{\sym \Delta}(T) \wr S_\ell$ in product action.
        \end{itemize}
    \end{defn}
    \pause

    \begin{lemma}
        $N_{\sym \Delta ^ \ell}( T ^ \ell ) = N_{\sym \Delta}(T) \wr S_\ell$.
    \end{lemma}
\end{frame}

\note[itemize]
{
\item
PA: an example is our running example!!
\item
\"Uberleitung to key idea, sketch on black board!!: \\
$\soc G \text{ char } G$ \\
$\soc G \unlhd N(G)$ \\
Thus contained in normaliser of socle
}

\begin{frame}{The Key Idea ...}
    \centering
    {\Large
    Construct $N_{\sym \Omega}(\soc G)$!
    }
\end{frame}

\begin{frame}{... And Why It Works ...}
    \begin{lemma}
        Let $G \leq \sym \Omega$ be primitive of type PA.
        Then
        \vspace{-0.5em}
        \[
            [ N_{\sym \Omega}(\soc G) : \soc G ]
            \leq \sqrt n \cdot 2 ^ {\log n \log \log n}.
        \]
    \end{lemma}
    \pause

    \begin{lemma}
        Let $G = \gen X \leq \sym \Omega$ be primitive of type PA.
        Furthermore let a generating set for
        $N_{\sym \Omega}(\soc G)$ be known.
        \\
        \pause
        Then $N_{\sym \Omega}(G)$ can be computed in time
        \vspace{-0.5em}
        \[
            O(n ^ 3 \cdot 2 ^ {2 \log n \log \log n} \cdot \abs X).
        \]
    \end{lemma}
\end{frame}

\note{
Explain:
\\
$\abs{\out T} \leq \sqrt n$
\\
$\abs {S_\ell} \leq \ell ^ \ell$
}

\begin{frame}{... And How To Do It}
    Compute:
    \[
        \soc G \circlearrowright \Omega
        \iso
        T ^ \ell \circlearrowright \Delta ^ \ell
    \]
\end{frame}

\note[itemize]{%
\item
equal and not only isomorphic
\item
PA WP is a very very special group!
}


\section{The Category of Permutation Groups}

\section{Constructing the Normaliser of the Socle}

\section{What To Take Away}

\end{document}
