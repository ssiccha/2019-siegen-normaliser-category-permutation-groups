 \documentclass{beamer}
%\documentclass[handout]{beamer}
\usetheme{metropolis}           % Use metropolis theme

 \setbeameroption{show notes}
%\setbeameroption{show notes on second screen}

% Packages
\usepackage{xcolor}
\usepackage{multirow}
% Maths
\usepackage{amsthm}

% Package Settings
% AMSTHM
\theoremstyle{plain}
\newtheorem{thm}{Theorem}[section]
\newtheorem{mylemma}[thm]{Lemma}
\newtheorem{cor}[thm]{Corollary}
\newtheorem{rem}[thm]{Remark}
\newtheorem{alg}[thm]{Algorithm}

\theoremstyle{definition}
\newtheorem{defn}[thm]{Definition}

\input{./abbreviations.tex}

\title{Normalisers in Quasipolynomial Time and \\
the Category of Permutation Groups}
\date{May 9, 2019}
\author{Sergio Siccha}
\institute{Lehr- und Forschungsgebiet Algebra, RWTH Aachen}

\begin{document}
\maketitle


\begin{frame}{Conventions}
\begin{itemize}
\setlength\itemsep{1em}
\item
$\log := \log_2$.
\pause
\item
All groups and sets are finite!
\pause
\item
$\Omega, \Delta$ denote sets,
$G, H, T$ denote groups.
\pause
\begin{itemize}
    \item
    $T$ \emph{always} denotes a finite non-abelian simple group.
    \pause
    \item
    If $T \leq \sym \Delta$ it acts transitively and non-regularly on $\Delta$.
    \pause
\end{itemize}
\item
Functions act from the left $f(x)$
but groups from the right: $\alpha ^ g = g(\alpha)$.
\end{itemize}
\end{frame}

\section{Introduction}
\begin{frame}{Goal}
    \begin{thm}
        Let $G = \gen{X} \leq \sym \Omega$ be a primitive group
        of {\color{blue} PA type}.
        The normaliser
        $N_{\sym \Omega}(G)$
        can be computed in {\color{blue} quasipolynomial} time
        $O(n ^ 3 \cdot 2 ^ {2 \log n \log \log n} \cdot \abs X)$.
    \end{thm}

    \vspace{1em}
    \pause
    Joint work with Prof. Colva Roney-Dougal.
\end{frame}

\begin{frame}{Recursion for Normalisers}
    \begin{center}
        \hspace{-5em}
        \begin{tabular}{r c}
            & Intransitive
            \\
            Mun See Chang & $\updownarrow$
            \\
            & Transitive
            \\
            & $\updownarrow$
            \\
            & Primitive
            \\
            Me & $\updownarrow$
            \\
            & Simple
        \end{tabular}
    \end{center}
\end{frame}


\section{Complexity and Computational Group Theory}
\begin{frame}{Complexity Classes}
    We use big $O$ notation.
    \pause
    \\[2em]
    \hspace{2em}
    \begin{tabular}{l l}
    Polynomial Time:
    &
    $f \in O(n ^ c)$
    \\
    \pause
    Quasipolynomial Time:
    &
    $f \in 2 ^ {O((\log n) ^ c)}$
    \\
    \pause
    Simply Exponential Time:
    &
    $f \in 2 ^ {O(n)}$
    \\
    \pause
    Exponential Time:
    &
    $f \in 2 ^ {O(n ^ c)}$
    \end{tabular}
\end{frame}

\note[enumerate]{
\item
properly explain why quasipolynomial is not polynomial!
\item
changing input size to $\log n$ jumps up two classes
\item
$A$ is easier than $B$
\\ OR \\
$A$ can be embedded into $B$
}

\begin{frame}{Complexity Overview}
    \begin{tabular}{l l}
    \only<4->{%
        \underline{Simply Exponential:}
    }
    \only<-3>{\hphantom{%
        \underline{Simply Exponential:}
    }}
    &
    \\
    &
    \only<4->{%
    Normaliser
    }
    \only<-3>{\hphantom{%
    Normaliser
    }}
    \\[1em]
    \only<2->{%
    \underline{Quasipolynomial:}
    }
    \only<-1>{\hphantom{%
    \underline{Quasipolynomial:}
    }}
    &
    \\
    &
    \only<3->{%
    String-Iso, Intersection, Centraliser
    }
    \only<-2>{\hphantom{%
    String-Iso, Intersection, Centraliser
    }}
    \\[2em]
    &
    \only<2->{%
    Graph-Iso
    }
    \only<-1>{\hphantom{%
    Graph-Iso
    }}
    \\[1em]
    \underline{Polynomial:}
    &
    \\
    &
    Base \& SGS, Composition Series,
    Socle
    \\
    \end{tabular}
\end{frame}

\note{
FIXME: USE UNCOVER OR ONLY ALTERNATIVE
https://tex.stackexchange.com/questions/13793/beamer-alt-command-like-visible-instead-of-like-only
}

\begin{frame}{Normaliser and Subproblems}
    \begin{tabular}{l l}
        \multicolumn{2}{c}
        {\underline{Simply Exponential} \hspace{4em} $~$}
        \\[0.5em]
        \multicolumn{2}{c}
        {Normalisers of arbitrary groups}
        \\[2.5em]
        \pause

        \multirow{2}{*}{\parbox{0.5\linewidth}{
            \underline{Polynomial}
            \\[0.5em]
            Normalisers of groups with \\ restricted composition factors
            \\[0.5em]
            Normalisers of simple groups
        }}
        \pause
        &
        \multirow{2}{*}{\parbox{0.5\linewidth}{
            \underline{Quasipolynomial}
            \\[0.5em]
            {Normalisers of primitive \\ groups}
        }}
    \end{tabular}
    \vfill
\end{frame}

\note{
Now also explain why we're not precise about whether $N_{sym \Omega}(G)$ or
$N_H(G)$.
}


\section{PA Type Groups and How To Normalise Them}
\begin{frame}{Fundamentals}
    \begin{defn}
        Let $G \leq \sym \Omega$ be transitive.
        $G$ is called \emph{imprimitive} if there exists a non-trivial
        $G$-invariant partition of $\Omega$.
        Otherwise it is called \emph{primitive}.
    \end{defn}
    \pause

    \begin{defn}
        Let $G \leq \sym \Omega$ and $H \leq \sym \Delta$ be permutation groups.
        We call a pair $(f, \varphi)$ with
        $f \from \Omega \to \Delta$
        and
        $\varphi \from G \to H$
        a \emph{permutation isomorphism}
        if
        \pause
        for all
        $g \in G$ and $\alpha \in \Omega$ holds
        $f(\alpha ^ g) = f(\alpha) ^ {\varphi(g)}$.
    \end{defn}
\end{frame}

\note[itemize]{
\item
Explain perm iso:
\begin{itemize}
    \item
    Relabel points and how to map $G \to H$ accordingly.
    \item
    $G, H \leq \sym \Omega$ perm iso iff. exists $\sigma \in \sym \Omega$ with
    $G ^ \sigma = H$.
    \item
    $f \from \Omega \iso \Delta$ induces unique
    group hom
    $\sym \Omega \iso \sym \Delta$.
\end{itemize}
\item
perm iso can be extended to the whole symmetric group
\\
$\leadsto$
The normaliser of $G$ over $\Omega$ is mapped to the normaliser of $H$ over
$\Delta$.
}

\begin{frame}{Socles}
    \begin{defn}
        Let $G$ be a group. The \emph{socle} of $G$, denoted $\soc G$,
        is the group generated by all minimal normal subgroups of $G$.
    \end{defn}
    \pause

    \begin{thm}
        The socle of a primitive group is characteristically simple.
    \end{thm}
    \pause

    \begin{thm}[O'Nan-Scott]
        Let $G \leq \sym \Omega$ be primitive.
        All possible permutational isomorphism types of $\soc G$ and
        $N_{\sym \Omega}(\soc G)$ are known.
    \end{thm}
\end{frame}

\begin{frame}{Wreath Products (1)}
    \begin{defn}
        Let $H \leq \sym \Delta$ and $K \leq S_\ell$.
        $K$ acts on the components of $H ^ \ell$.
        \pause
        The semidirect product
        $H \wr K = H ^ \ell \rtimes K$
        is called the \emph{wreath product of $H$ with $K$}.
        \pause
        $H ^ \ell$ is called the \emph{base group}.
        \pause
        $K$ is called the \emph{top group}.
    \end{defn}
    \vspace{1em}
    \pause

    \begin{thm}
        $\aut( T ^ \ell ) \cong \aut(T) \wr S_\ell$
    \end{thm}
\end{frame}

\note{
    Make sure to explain intuition behind wreath products
}

\begin{frame}{Wreath Products (2)}
    \begin{defn}
        Let $H \leq \sym \Delta$ and $K \leq S_\ell$.
        The base group $H ^ \ell$ acts component-wise on $\Delta ^ \ell$.
        The top group $K$ acts on the components of $\Delta ^ \ell$.
        \pause
        This yields an action of $H \wr K$ on $\Delta ^ \ell$
        which we call the \emph{product action of $H \wr K$}.
        \\
        \pause
        We call the permutation group on $\Delta ^ \ell$ induced by $H \wr K$
        the \emph{product action wreath product of $H$ wirh $K$} and
        also denote it by $H \wr K$.
    \end{defn}
    \vspace{1em}
    \pause

    \begin{thm}
        Let $H \leq \sym \Delta$ and $K \leq S_\ell$.
        $H \wr K$ in product action is primitive
        if and only if
        $H$ is primitive and non-regular and $K$ is transitive.
    \end{thm}
\end{frame}

\note{
Explain WP actions via

base

top
}

\begin{frame}{The AS Type}
    \begin{defn}
        Let $G \leq \sym \Omega$ be a primitive group.
        \\
        We say $G$ is a group of \emph{AS type} if
        $\soc G = T$ is non-abelian simple
        and $G$ is almost simple.
    \end{defn}
\end{frame}

\note{
    Explain AS via Normaliser of $T$.
}

\begin{frame}{The PA Type}
    \begin{defn}
        Let $G \leq \sym \Omega$ be a primitive group.
        \\
        We say $G$ is a group of \emph{PA type} if it is permutation
        isomorphic to a group $\myhat G \leq \sym \Delta ^ \ell$ with:
        \vspace{-0.5em}
        \pause
        \begin{itemize}
            \item
            $\soc \myhat G = T ^ \ell$,
            \pause
            \item
            $\myhat G \leq N_{\sym \Delta}(T) \wr S_\ell$.
        \end{itemize}
    \end{defn}
    \pause

    \begin{lemma}
        $N_{\sym \Delta ^ \ell}( T ^ \ell ) = N_{\sym \Delta}(T) \wr S_\ell$.
    \end{lemma}
\end{frame}

\note[itemize]
{
\item
AS: explain relationship with normaliser $N_{\sym \Delta}(T)$.
\item
\"Uberleitung to key idea: \\
$\soc G \text{ char } G$ \\
$\soc G \unlhd N(G)$ \\
Thus contained in normaliser of socle
}

\begin{frame}{The Key Idea ...}
    \centering
    {\Large
    Construct $N_{\sym \Omega}(\soc G)$!
    }
\end{frame}

\begin{frame}{... And Why It Works ...}
    \begin{lemma}
        Let $G \leq \sym \Omega$ be primitive of type PA.
        Then
        \vspace{-0.5em}
        \[
            [ N_{\sym \Omega}(\soc G) : \soc G ]
            \leq \sqrt n \cdot 2 ^ {\log n \log \log n}.
        \]
    \end{lemma}
    \pause

    \begin{lemma}
        Let $G = \gen X \leq \sym \Omega$ be primitive of type PA.
        Furthermore let a generating set for
        $N_{\sym \Omega}(\soc G)$ be known.
        \\
        \pause
        Then $N_{\sym \Omega}(G)$ can be computed in time
        \vspace{-0.5em}
        \[
            O(n ^ 3 \cdot 2 ^ {2 \log n \log \log n} \cdot \abs X).
        \]
    \end{lemma}
\end{frame}

\note{
Explain:
\\
$\abs{\out T} \leq \sqrt n$
\\
$\abs {S_\ell} \leq \ell ^ \ell$
}

\begin{frame}{... And How To Do It}
    Compute:
    \[
        \soc G \circlearrowright \Omega
        \iso
        T ^ \ell \circlearrowright \Delta ^ \ell
    \]
    \pause

    Then:
    \[
    \begin{array}{ccll}
        G                                   & \lhook\joinrel\longrightarrow
        & N_{\sym \Delta ^ \ell}(T ^ \ell)
        \pause
                                            & = ~ N_{\sym \Delta}(T) \wr S_\ell
        \\[1em]
        \pause
        N_{\sym \Omega}(\soc G)
        &
        \xleftarrow{\,\raisebox{-1pt}{\ensuremath{\scriptstyle{\sim}}}\,}
        & N_{\sym \Delta ^ \ell}(T ^ \ell)
    \end{array}
    \]
\end{frame}

\note[itemize]{%
\item
equal and not only isomorphic
\item
PA WP is a very very special group!
}


\section{The Category of Permutation Groups}
\begin{frame}{Permutation Homomorphisms (1)}
   \begin{defn}
       Let $G \leq \sym \Omega$ and $H \leq \sym \Delta$ be permutation groups.
       A tuple
       $(f, \varphi)$
       with
       map $f \from \Omega \to \Delta$
       and
       group hom.
       $\varphi \from G \to H$
       is called a
       \emph{%
       permutation hom. from $(G, \Omega)$ to $(H, \Delta)$
       }
       if for all $g \in G$ holds
       \[
           FIXME ~COMMUTING DIAGRAM
       \]
   \end{defn}
\end{frame}

\begin{frame}{Permutation Homomorphisms (2)}
    \begin{lemma}
        Let $G \leq \sym \Omega$ and $f \from \Omega \to \Delta$.
        There exist a group $H \leq \sym \Delta$
        and a group hom. $\varphi \from G \to H$ such that
        $(f, \varphi)$ is a permutation hom.
        if and only if
        \[
            \sset{ f ^ {-1}(\set{x}) }{ x \in \im f }
        \]
        is $G$-invariant.
    \end{lemma}
\end{frame}

\begin{frame}{Permutation Homomorphisms (3)}
    FIXME EXAMPLE

    \begin{rem}
        Let $G \leq \sym \Omega$ and $H \leq \sym \Delta$.
        $f \from \Omega \onto \Delta$ uniquely determines,
        if it exists,
        a group hom. $\varphi \from G \to H$ such that
        $(f, \varphi)$
        is a permutation hom.
    \end{rem}
\end{frame}

\begin{frame}{PermGrp}
    \begin{defn}
        FIXME Define $\permgrp$.
    \end{defn}
\end{frame}

\note{equiv to cat of $(G, \Omega, \rho)$}

\begin{frame}{Product in PermGrp}
    \begin{lemma}
        Let $G \leq \sym \Omega$ and $H \leq \sym \Delta$ be permutation
        groups.
        Then $(G \times H, \Omega \times \Delta)$ with
        $(p_1, \pi_1)$ and $(p_2, \pi_2)$ is a product in $\permgrp$.
    \end{lemma}
\end{frame}

\begin{frame}{Cartesian Decompositions}
    \begin{defn}
        Let $\mathcal C$ be a category and $X$ an object of $\mathcal C$.
        A family of morphisms $(f_i)_{i \in I}$ with
        $f_i \from X \to X_i$
        is called a
        \emph{cartesian decomposition of $X$}
        if
        \[
            \prod_{i \in I} f_i
            \from
            X
            \to
            \prod_{i \in I} X_i
        \]
        is an isomorphism.
    \end{defn}

    \begin{lemma}
        %Let $\mathcal C$ be a category and $X$ an object of $\mathcal C$.
        A family $(f_i)_{i \in I}$ is a cartesian decomposition of $X$
        if and only if
        $X$ with $(f_i)_{i \in I}$ is a product in $C$.
    \end{lemma}
\end{frame}

\begin{frame}{Homogeneous Cartesian Decompositions}
    \begin{defn}
        FIXME hom cartesian decomposition.
        For all $i,j \in I$ have $f_i(X) \cong f_j(X)$
    \end{defn}

    \begin{defn}
        FIXME strongly hom cartesian decomposition
        For all $i,j \in I$ have $f_i(X) = f_j(X)$
    \end{defn}

    $\Rightarrow$
    Compute a strongly homogeneous cartesian decomposition of $\soc G$!
\end{frame}

\begin{frame}{Combinatorial Cartesian Decompositions}
    FIXME LEAVE THIS FRAME OUT?
    \begin{defn}
        CCD
    \end{defn}

    \begin{lemma}
        unordered cd bijection CCD
    \end{lemma}

    \begin{thm}[Praeger, Schneider]
        $G$ leaves CCD invariant
        if and only if
        $G$ embeds into PA WP.
    \end{thm}
\end{frame}


\section{Constructing the Normaliser of the Socle}
\begin{frame}{The Algorithm - Input}
    \centering
    \large
    Let $G = \gen X \leq \sym \Omega$ be a primitive group of PA type.
    \\[1em]
    \pause
    Note that $T ^ \ell$ has \emph{exactly} $\ell$ minimal normal subgroups.
\end{frame}

\note{
$T ^ \ell$ has exactly one ``basis''!
\\
On next frame:
mention $i$ always stands for all $i$ from $1$ to $\ell$
}

\begin{frame}{The Algorithm - Str. Hom. Cartesian Decomposition}
    \begin{alg}
        $~$
        \\[-1em]
        \begin{itemize}
            \item
            $\soc G \pause ~ (= T_1 \times \ldots \times T_\ell)$.
            \pause
            \item
            minimal normal subgroups $T_i$ of $\soc G$.
            \pause
            \item
            complements $C_i$ of the $T_i$,
            \pause
            partitions $\Delta_i = \set{ \text{orbits of } C_i }$.
            \pause
            \item
            $Q_i \from \Omega \to \Delta_i,
            ~ \alpha \mapsto \alpha ^ {C_i}$
            \pause
            $~~\Rightarrow~~ \psi_i \from G \to T_i$.
            \pause
            \item
            $g_1, \ldots, g_\ell \in G$ such that $T_i ^ {g_i} = T_1$.
            \pause
            \item
            $R_i \from \Delta_i \to \Delta_1,
            ~ \delta \mapsto \delta ^ {g_i}$
            \pause
            $~~\Rightarrow~~ \rho_i \from T_i \to T_1$.
            \pause
            \item
            $P_i := R_i \circ Q_i \from \Omega \to \Delta_1$
            \pause
            $\hspace{0.5em}\Rightarrow~~ \varphi_i \from G \to T_1$.
        \end{itemize}
    \end{alg}
    \pause

    $((P_i, \varphi_i))_{i \leq \ell}$
    is strongly homogeneous cartesian decomposition.
\end{frame}

\begin{frame}{The Algorithm - Normaliser of Socle}
    \begin{itemize}
        \setlength\itemsep{\fill}
        \item
        $((P_i, \varphi_i))_{i \leq \ell}$
        is a strongly homogeneous cartesian decomposition of $\soc G$.
        \pause
        \item
        This yields
        $\soc G \circlearrowright \Omega
        \iso
        T ^ \ell \circlearrowright \Delta ^ \ell$.
        \pause
        \item
        Compute $N_{\sym \Delta}(T)$.
        \pause
        \item
        Construct $N_{\sym \Delta}(T) \wr S_\ell \leq \sym \Delta ^ \ell$.
        \pause
        \item
        Map back into $\sym \Omega$.
    \end{itemize}
    \pause
    $\leadsto N_{\sym \Omega}(\soc G)$.
\end{frame}

\note{
Compute $N_{S_\ell}(K)$ in simply exponential time.

$\ell \leq \log n \Rightarrow$ polynomial time.
}


\section{Summary}
\begin{frame}{What To Take Away}
    \begin{itemize}
        \item
        Category Theory makes (some) algorithms nicer.
        \pause
        \item
        For primitive groups of PA type
        we can construct the normaliser of the socle in polynomial time.
        \pause
        \item
        For primitive groups of PA type
        we can compute the normaliser in quasipolynomial (maybe even
        polynomial?) time.
    \end{itemize}
\end{frame}


\end{document}
