Normalisers in Quasi-Polynomial Time and the Category of Permutation Groups

Computing normalisers of permutation groups is considered a
notoriously hard problem, both in theoretical computer science and in
computational group theory.
In joint work with Colva Roney-Dougal, we use the classification of
finite primitive groups to show that normalisers of primitive groups
can be computed by a sub-exponential time algorithm.

In this talk I present my strategy for a sub-class of primitive
groups which yields both a theoretical quasi-polynomial time bound as
well as a substantial improvement to practical algorithms.
I will make a short detour to introduce the category of permutation
groups in order to present the realisation of that strategy in a
concise language, and if time permits also of the algorithm.
